\svnidlong
{$HeadURL: $}
{$LastChangedDate: $}
{$LastChangedRevision: $}
{$LastChangedBy: $}

\odachapter{Parallel computing in OpenDA using Threading and Java RMI}

\begin{tabular}{p{4cm}l}
\textbf{Contributed by:} & Nils van Velzen\\
\textbf{Last update:}    & \svnfilemonth-\svnfileyear\\
\end{tabular}

\section{Front-end models in \oda}
Front-end models are a special kind of models in \oda. A front-end model
implements the \oda Stochastic model interface but it does not implement any
model equations. A front-end model adds additional functionality on top of an
arbitrary OpenDA Stochastic model. This model is called the back end model. In
general, all methods of the front-end model are implemented by use of methods
of the back-end model.

The front-end models implement generic extension to \oda Stochastic models.

In this chapter we will give a description of two of these kind of front-end
models. The Thread Stochastic Model adds some parallelism to arbitrary \oda
stochastic models using threading. The RMI Stochastic Model allows models to be
run in a different process and optionally remotely on a remote computer.

\section{Parallelism using Threading}
\subsection{introduction}
Java has a good support for threading. Using threads, a program can make use of
all the available computational cores in a computer. Threading is therefore in
Java a simple and natural way to implement shared-memory parallelism.

Many data assimilation and calibration algorithms perform a (large) number of
model simulation steps
\begin{equation}
x\left(t+\Delta t \right)= M\left(x\left(t \right),u\left(t \right), p\right)
\end{equation}
for various state vectors $x\left(t \right)$ or parameters $p$. When these
model simulations are independent, we can compute them in parallel.

\subsection{Thread Stochastic Model}
\subsubsection{Basic usage and configuration}
\oda provides a front-end model that parallelizes the compute method of a
stochastic model using threading. This model can be used as a front-end to any
\oda model that implements the compute method in a thread safe way. The maximal
number of threads can be specified in order to limit the number of simultaneous
compute invocations.

The model factory of the threaded front-end model is
\begin{verbatim}
org.openda.models.threadModel.ThreadStochModelFactory
\end{verbatim}

The configuration file of the {\tt ThreadStochModelFactory} specifies
\begin{itemize}
\item maxThreads; the maximum number of simultaneous compute threads. This is typically set to the
	maximum number of cores that are available to the user (on a single computer).
\item numThreadGroups; the number of groups each having maxThreads treads. This is typically the number of 
      hosts that are used for parallel computing. This only makes sense when the Thread Stochastic Model is combined
      with the RMI Stochastic Model.
\item stochModelFactory; the configuration of the the back-end stochastic model factory.
\end{itemize}

A typical example of the {\tt ThreadStochModelFactory} configuration looks like
{\small
\begin{verbatim}
<threadConfigstoch>
   <maxThreads>8</maxThreads>
   <numThreadGroups>1</numThreadGroups>
   <stochModelFactory
      className="org.openda.models.lorenz.LorenzStochModelFactory">
      <workingDirectory>model</workingDirectory>
      <configFile>LorenzStochModel.xml</configFile>
   </stochModelFactory>
</threadConfigstoch>
\end{verbatim}
}

{\bf NOTE1:} The compute method of your model MUST be thread safe. Java models
will often be thread safe. Native models are in general not thread safe.\\

{\bf NOTE2:} The working directory for the back-end model is NOT relative to
the location of the {\tt ThreadStochModelFactory} configuration file but to the
"main" \oda configuration file.
%Typically you have your model configuration file(s) in a directory called {\tt
%model} relative main {\tt oda} file. When you place both (thread and back-end)
%model configuration files in the directory model you have to fill
%in \begin{verbatim} <workingDirectory>model</workingDirectory> \end{verbatim}
%in the thread model configuration file, like in our example file.

\subsubsection{advanced features}
The basic usage of thread stochastic model only includes specifying the max
number of threads and the back-end model factory. There are some additional
options that can improve the performance of your parallel application. These
advanced options are described in this section. The impact of these options
depend heavily on the used model and architecture.
\begin{itemize}
\item {\tt sleepTime} (integer) the number of miliseconds sleep between polling whether a thread running the model has finished. When set too small, the polling process can take a significant amount of CPU time slowing down the model runs. The default value of 10 (ms)will work ok for most models. 
\item {\tt cashState} (boolean) if set "true", a parallel nonblocking getState
  method is invoked after a parallel compute is completed. This parallel
  getState method is invoked on a new thread and not limited by the maxThread
  option. In this way, the data assimilation algorithm already receives a copy
  of the state while other model instances are still computing. This can be
  useful when the time of the getState method is dominated by IO (file or
  network). The state is stored internally until the getState method is
  invoked.

\item{\tt nonBlockingAxpy} (boolean) if set "true", a invocation of the
  axpyState method will be non-blocking. This option might improve performance
  when the time of the axpyState method is dominated by IO (file or network).
\end{itemize}

The XML configuration using these three options looks like:
{\small
\begin{verbatim}
<threadConfig>
   <maxThreads>2</maxThreads>
   <sleepTime>50</sleepTime>
   <cashState>true</cashState>
   <nonBlockingAxpy>true</nonBlockingAxpy>
   <stochModelFactory
    className="org.openda.models.rmiModel.RmiClientStochModelFactory">
      <workingDirectory>./stochModel</workingDirectory>
      <configFile>RmiStochModel2.xml</configFile>
      </stochModelFactory>
</threadConfig>
\end{verbatim}
}

\section{Remote Method Invocation}

\subsection{Introduction}
RMI (Remote Method Invocation) is a way in java to make a connection between
objects in various virtual machines (executables). In \oda we provide a
front-end model that enables us to create and use stochastic models on various
executables and computers. Note that RMI allows us to use multiple processes in
our application but that computations are not automatically in parallel.

\subsection{RMI Stochastic Model}
The RMI provides the (parallel) computation of model simulation steps in a
different virtual machine as the data assimilation method or the model
calibration method. This serves two goals:
\begin{itemize}
\item multiple model time steps can be computed in parallel (in combination
  with the Thread model front-end)
\item the model can runs on a dedicated machine (server) where the data
  assimilation method runs locally.
\end{itemize}

The \oda application running the data assimilation algorithm or model
calibration algorithm is called the client. This executable will make use of
one or more servers, possibly running on remote machines.

The RMI front-end model factory is
\begin{verbatim}
org.openda.models.rmiModel.RmiClientStochModelFactory
\end{verbatim}
The configuration file of the {\tt RmiClientStochModelFactory} specifies
\begin{itemize}
\item serverAddress; The names of the computers running the servers. There are
  three ways of configuration possible:
  \begin{itemize}
  \item empty; all servers run on the local host
  \item single machine name; all servers run on this machine
  \item names of all computers, comma separated. Note that the number of
    computer names must be the same as the number of specified factoryIDs. The
    same computer name can be used multiple times when multiple servers run on
    that computer.
  \end{itemize}
\item factoryID; The unique IDs of all the servers, comma separated,
  implementing the remote models and model factories.
\item stochModelFactory; the configuration of the the back-end stochastic model
  factory. {\bf Important note}: The {\tt workingDirectory} and {\tt
    configFile} are communicated to the server processes. The use of relative
  paths is only possible when the servers are started from an appropriate
  location. Full paths are therefore often a better choice.
\end{itemize}

A typical example of a {\tt RmiClientStochModelFactory} configuration file is
{\small
\begin{verbatim}
<rmiConfig>
   <serverAddress></serverAddress>
   <factoryID>IRmiIStochModel_1,IRmiIStochModel_2</factoryID>
   <stochModelFactory
      className="org.openda.models.lorenz.LorenzStochModelFactory">
      <workingDirectory>./model</workingDirectory>
      <configFile>LorenzStochModel.xml</configFile>
   </stochModelFactory>
</rmiConfig>
\end{verbatim}
}

\subsection{RMI Stochastic Model Server}
In the previous section we have explained how to use the RMI front-end model
from the client side. The other side are the server processes. The java class
that implements the server is {\tt org.openda.models.rmiModel.Server}.

Before the servers can be started a special daemon process must be started {\tt
  rmiregistry} on each host computer. This daemon process allows RMI clients
and servers to connect. The following example shows a shell script that starts
and initialises two servers on local host. {\small
\begin{verbatim}
#!/bin/sh

# append all jars in opendabindir to java classpath
for file in $OPENDADIR/*.jar ; do
   if [ -f "$file" ] ; then
       export CLASSPATH=$CLASSPATH:$file
   fi
done

rmiregistry &
echo wait 5 seconds for rmiregistry to start and initialize
sleep 5

# start the server with a non-default factory ID
echo starting server
java  -Djava.rmi.server.codebase=file:///$OPENDADIR/ \
      org.openda.models.rmiModel.Server IRmiIStochModel_1 0 2&
java  -Djava.rmi.server.codebase=file:///$OPENDADIR/ \
      org.openda.models.rmiModel.Server IRmiIStochModel_2 1 2&
\end{verbatim}
} Note the {\tt -Djava.rmi.server.codebase=file:\/\/\/\$OPENDADIR} option. This
specifies the starting point to the classes that are loaded by the server. The
two arguments have the following meaning:
\begin{enumerate}
\item (string) ID of this server
\item (int) the sequence number of this server (0,...,number of servers -1)
\item (int) total number of servers
\end{enumerate}
The last two numbers are used to configure the DistributedCounter as will be
explained in the following section.

\subsection{InfiniBand and multiple network cards}
Some nodes have multiple network connections. On supercomputers you will
typically see that nodes both have an ethernet and an InfiniBand connection. In
this case the computer will have two ip addresses and probably two names. The
RMI server will use the network connection that corresponds to the \$HOSTNAME
of the node. This is typically the ethernet connection and not the fast
InfiniBand we would like to use. The java option
\begin{verbatim}
-Djava.rmi.server.hostname="InfiniBandHostname"
\end{verbatim}
can be passed to Java when starting the servers. In this way, the servers will
use the InfiniBand connection for communication to the algorithm process.


\subsection{Random numbers and DistributedCounter}
When running in parallel you sometimes need to generate "unique" numbers in the
model instances. E.g. when each model instance generates (or reads) its own
input and output files and when you want to use some numbering to distinguish
them. In order to globally define unique counters we have introduced the
DistributedCounter class in OpenDa. When used in a sequential run, a
DistributedCounter instance is just a integer counter 0,1,2,3,etc. In parallel
runs, the DistributedCounter instance will generate a sequence $i_p+j n_p$ for
j=$0,1,....$ where $i_p$ denotes the process number and $n_p$ the total number
of processes.

A similar issue is the generation of pseudo random numbers. We must be very
careful when we generate random numbers in a parallel environment. If the
random generators in various threads of processes are all initialised with the
same initial seed, we will get wrong results since the same pseudo random
numbers will be drawn on the various processes. The DistributedCounter can be
used to initialise pseudo random generators. When the user uses one of the
noise models from \oda this should work correct in parallel since these random
generators make use of the DistributedCounter.


\subsection{Serialization}
All arguments (classes) that are passed between the remote model instances need
to "extends Serializable". If this is not the case Java cannot pack the content
and send the object to an other processes. Many of the relevant classes in \oda
are already Serializable but user provided classes are often not. When a class
is not Serializable you will get a run-time error when you try to run in
parallel. Fortunately, this is often easy to fix by adding "extends
Serializable" to these classes (and subclasses).

\subsection{Limitations}
The RMI front-end model is work in progress. There are therefore some
limitations including
\begin{itemize}
\item Only java implementations of (Tree-)Vectors can be used because the
  serialization of native objects is not yet implemented
\item We have not done any work/testing on client/server security issues but we
  did not encounter any problems on the HPC computer systems we have used.
  There might be some more work to be done for using these tools in more
  secured environments.
\end{itemize}

\section{Parallel computing with multiple executables}
The Thread Stochastic model allows parallel computation of model time steps.
This parallelism is limited to a single executable and maximized by the amount
of cores of a single computer. The RMI Stochastic model allows multiple
executables to be used for the model computations. But The RMI model does not
by itself support parallelism. However when both front-end models are combined
we are able to compute model time steps in parallel using multiple processes
and multiple computers. The Thread model is then used as a frond-end to the RMI
model which is the front-end of the physical model.

Models that are not thread safe can be parallelized by combining both front end
models because at the server side no time steps are computed in parallel.
